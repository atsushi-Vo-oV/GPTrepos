
\begin{frame}{Infinite-dimensional HiPPO}
\[
    \frac{\partial h}{\partial t} 
    = A^{\mathrm{hippo}} h + B^{\mathrm{hippo}} u
\]
を要素ごとに書き直すと
\[
\begin{split}
    \frac{\mathrm{d} h_i}{\mathrm{d} t} =& 
    - \sqrt{2i+1} \left(
    \textcolor{orange}{
        \frac{h_0}{2} 
        + \sum_{l=1}^{i - 1} \sqrt{2l +1} h_l
        + \sqrt{2i +1} \frac{h_i}{2}
    }\right) 
    - \frac{h_i(t)}{2} \\
    & + \sqrt{2i+1} \left(
    - \frac{1}{2} h_0
    + u(t)
    \right)
\end{split}
\]
ここから、\textcolor{orange}{橙部分}が台形積分近似であることから、
\[
\begin{split}
    \frac{\partial \Tilde{h}}{\partial t} (t)(x) 
    & = - \sqrt{2x+1} \textcolor{orange}{\int_0^x \sqrt{2\xi + 1} \Tilde{h}(t)(\xi) d\xi }
    - \frac{1}{2} \Tilde{h}(t)(x)
    \\
    & + \sqrt{2x+1} \left(-\frac{1}{2}\int_0^t e^{-(t-s)} u(s) ds + u(t) \right) 
\end{split}    
\]
が導ける。
\end{frame}
