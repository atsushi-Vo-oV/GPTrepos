
\begin{frame}{Infinite-dimensional HiPPO}
HiPPO 行列を用いたLSSLは次のように解釈できる。
\begin{screen}
\textcolor{purple}{\textbf{Theorem:}}
HiPPO 行列による$\mathbb{R}^{N + 1}$上のLSSLの微分方程式
\[
    \frac{\partial h}{\partial t} = A^{\mathrm{hippo}} h + B^{\mathrm{hippo}}u
\]
は状態空間を\( L^2([0, N]) \)とした微分方程式
\[
\begin{split}
    \frac{\partial \Tilde{h}}{\partial t} (t)(x) 
    & = - \sqrt{2x+1} \int_0^x \sqrt{2\xi + 1} \Tilde{h}(t)(\xi) d\xi 
    - \frac{1}{2} \Tilde{h}(t)(x)
    \\
    & + \sqrt{2x+1} \left(-\frac{1}{2}\int_0^t e^{-(t-s)} u(s) ds + u(t) \right) 
\end{split}    
\]
を台形公式を用いて$\xi$方向に離散化したものと一致する。

ただし、$d\xi \approx \Delta \xi = 1$、すなわち$h_i \approx \Tilde{h}(i)$である。
\end{screen}

\end{frame}
