
\begin{frame}{Infinite-dimensional HiPPO}
\begin{screen}
\textcolor{blue}{\textbf{Definition:}}
\(L^2([0,N])\)上の作用素\(\mathcal{F}_{\chi, \omega}, \mathcal{G}_{\chi}\)をInfinite-dimensional HiPPOという。
\end{screen}
この\(\mathcal{F}_{\chi, \omega}, \mathcal{G}_{\chi}\)を使った状態空間モデルについて、次のことがわかる。

\begin{screen}
\textcolor{purple}{\textbf{Theorems:}}
    $v$はLipschitz連続であるとする。 
    
    このとき、Infinite-dimensional HiPPOを用いたLSSL
    \[
    \frac{\partial \Tilde{h}}{\partial t} (t) 
    = \mathcal{F}_{\chi_0, \frac{1}{2}}[\Tilde{h}(t)] + \mathcal{G}_{\chi_0}[v(t)].
    \]
    の強解は次のように書ける。
    \small
    \[
        y(t) = \int_0^t \left(
        e^{- \omega (t - \tau)} \int_0^N c(x) \chi(x) J_0\left(2\sqrt{t - \tau}\sqrt{\int_{0}^x \left|\chi(s)\right|^2 \mathrm{d} s}\right) \mathrm{d} x
        \right) 
        v(\tau)  \mathrm{d} \tau 
    \]
\end{screen}

\end{frame}