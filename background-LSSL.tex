
\begin{frame}{Background}

\ftheme{State Space Model, LSSL}


\begin{itemize}
    \item 次で表される変換\(u \to y\)を\eqatn{状態空間モデル(SSM)}という。
    \item 特にこの変換を深層学習の層として用いたものを\eqatn{Linear State Space Layer(LSSL)}という。
\end{itemize}

\begin{itembox}{\eqatn{SSM, LSSL}}
    \noindent
    \begin{gather*}
        \frac{dh}{dt}(t) = Ah(t) + Bu(t) \\
        y(t) = Ch(t) \\ 
        h(0) = 0
    \end{gather*}
\end{itembox}
{
\footnotesize
ただし、
\(u:[0,T] \to \mathbb{B}_\mathrm{in}\),
\(h:[0,T] \to \mathbb{B}_\mathrm{out}\), 
\(A \in \mathcal{L}(
    \mathbb{B}_{\mathrm{state}},
    \mathbb{B}_{\mathrm{state}})
\),
\(
B \in \mathcal{L}(\mathbb{B}_{\mathrm{in}}, \mathbb{B}_{\mathrm{state}})
\),
\(C \in \mathcal{L}(\mathbb{B}_{\mathrm{out}}, \mathbb{B}_{\mathrm{state}})\)。
\(\mathbb{B}_*\)はバナッハ空間を意味し、\(\mathcal{L}(\mathbb{B}_1, \mathbb{B}_2)\)は\(\mathbb{B}_1\)から\(\mathbb{B}_2\)への線形写像全体を表す。
}
\end{frame}