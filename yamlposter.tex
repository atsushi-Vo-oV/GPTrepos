\documentclass[final]{beamer}

\usepackage{mathrsfs} 
\usepackage[orientation=portrait,size=a0,scale=1.4]{beamerposter}
\usepackage{amsmath,amssymb,amsfonts}
\usepackage{graphicx}
\usepackage{tikz}
\usepackage{booktabs}
\usepackage{multicol}
\usepackage{bbm}
\usepackage{caption}
\usepackage{CJKutf8}

\newcommand{\inner}[3]{
    \langle #1, #2 \rangle _{#3}
}

\usepackage[dvipsnames]{xcolor} 
\usepackage{ascmac}

% Custom colors and styles
\definecolor{colortheme1}{RGB}{154, 111, 192}
\definecolor{colortheme2}{RGB}{111, 132, 192}
\usetheme{Rochester}
\setbeamercolor{background canvas}{bg=white}
\setbeamercolor{structure}{fg=colortheme1}
\setbeamercolor{block title}{fg=white,bg=colortheme1}
\setbeamercolor{block body}{fg=black,bg=white}

\newenvironment{attbox}[1]{
\begin{itembox}{{\color{black} \bf \textbf{#1}}}
}{
\end{itembox}
}


% Title style
\title{「HiPPO 行列を用いた状態空間モデル」の無限次元化}
\author{\underline{Atsushi Takabatake} \and Takaharu Yaguchi}
\institute{Department of Mathematics, Kobe University}


% 数式のフォントを変更しない
\usefonttheme{professionalfonts} 

\begin{document}
\begin{CJK}{UTF8}{min}
\begin{frame}{
    \centering
    {\usebeamerfont{title}\huge \inserttitle}
}


\begin{beamercolorbox}[wd=\paperwidth,ht=0.01\paperheight,center]{structure}
  % \usebeamerfont{title}\huge\textcolor{black}{\inserttitle}
  \Large\textcolor{black}{\insertauthor \and -\and \insertinstitute}
\end{beamercolorbox}



\begin{columns}[T]
\begin{column}{0.48\linewidth}

\begin{block}{Abstract}
時系列データ処理において、長距離の依存関係を扱える軽量なモデルとして、HiPPOを用いたLSSLが提案されている。この研究では、その HiPPO行列に対応する無限次元化(無限次元ベクトル空間上の線形作用素)を提案し、その作用素によるLSSLの解について明示的解を与える。
\end{block}

\begin{block}{Background - LSSL}
作用素$A, B, C$を用いた次の変換 $u \mapsto y$ を状態空間モデル(State Space Model, SSM)という\textsuperscript{*1} \cite{LSSL}。
\begin{attbox}{SSM, LSSL}
\begin{align*}
\frac{dh}{dt}(t) &= Ah(t) + Bu(t) \\
y(t) &= Ch(t) \\ 
h(0) & = 0
\end{align*} 
\end{attbox}
特に、この変換を(時刻方向に離散化して)機械学習の層として用いたものを\textbf{LSSL}(Linear State Space Layer)と呼ぶ\textsuperscript{*2}。

このLSSLの解は形式的に、次のように書ける。
\begin{screen}
    \[
    y(t)=\int_0^t
    C\exp\left((t - \tau) A\right)B
    u(\tau)
    \mathrm{d} \tau
    \]
\end{screen}
これは$u$がスカラー関数の時、LSSLの出力が
\(
K(t) = 
C\exp\left(t A\right)B1
\)
なる関数と\(u\)の畳み込みで表せられることを意味する。ここで、上記の解は作用素の指数関数を用いており、形式的解であることに注意。

{
\footnotesize
(*1)
細かく書くなら
\(u:[0,T] \to \mathbb{B}_\mathrm{in}\),
\(h:[0,T] \to \mathbb{B}_\mathrm{out}\), 
\(A \in \mathcal{L}(
    \mathbb{B}_{\mathrm{state}},
    \mathbb{B}_{\mathrm{state}})
\),
\(
B \in \mathcal{L}(\mathbb{B}_{\mathrm{in}}, \mathbb{B}_{\mathrm{state}})
\),
\(C \in \mathcal{L}(\mathbb{B}_{\mathrm{out}}, \mathbb{B}_{\mathrm{state}})\)。
\(\mathbb{B}_*\)はバナッハ空間を意味し、\(\mathcal{L}(\mathbb{B}_1, \mathbb{B}_2)\)は\(\mathbb{B}_1\)から\(\mathbb{B}_2\)への線形写像全体を表す。
}

{
\footnotesize
(*2) LSSLでは$\mathbb{B}_\mathrm{in} = \mathbb{B}_\mathrm{out} = \mathbb{R}$, $\mathbb{B}_\mathrm{state} = \mathbb{R}^N$としている。これを$d_{\mathrm{model}}$次並列に用いる。
}


\end{block}

\begin{block}{Background - HiPPO Matrix}
HiPPO 行列は次で定義される\cite{HiPPO}:
\begin{attbox}{HiPPO Matrix}
\[
A^{\mathrm{hippo}}_{ij} =
-
\begin{cases}
(2i+1)^{1/2}(2j+1)^{1/2} & (i > j)\\
i + 1 & (i = j)\\
0 & (i < j)
\end{cases}
\]
\[
    B_{i}^{\mathrm{hippo}} = \sqrt{2i + 1}.
\]

\end{attbox}
このHiPPO行列をLSSLに使うと、機械学習の層としての性能が上がることが(実験的に)知られている。
\end{block}




\begin{block}{Infinite Dimensional HiPPO}
HiPPOを用いたLSSLは状態空間として$L^2$を用いたものから離散化したものであると解釈できる。以下の定理はこれを述べている。
\begin{screen}
\textcolor{purple}{\textbf{Theorem:}}
HiPPO 行列による$\mathbb{R^N}$上のLSSLの微分方程式
\[
    \frac{\partial h}{\partial t} = A^{\mathrm{hippo}} h + B^{\mathrm{hippo}}u
\]
は状態空間を\( L^2([0, N]) \)とした微分方程式
\[
\begin{split}
    \frac{\partial \Tilde{h}}{\partial t} (t)(x) 
    & = - \sqrt{2x+1} \int_0^x \sqrt{2\xi + 1} \Tilde{h}(t)(\xi) d\xi 
    - \frac{1}{2} \Tilde{h}(t)(x)
    \\
    & + \sqrt{2x+1} \left(-\frac{1}{2}\int_0^t e^{-(t-s)} u(s) ds + u(t) \right) 
\end{split}    
\]
を台形公式を用いて$\xi$方向に離散化したものと一致する。

ただし、$d\xi \approx \Delta \xi = 1$、すなわち$h_i \approx \Tilde{h}(i)$である。
\end{screen}
この$L^2$上の微分方程式は次のように状態空間の形に書きなおせる\textsuperscript{*3}。
\begin{screen}
\[
    \frac{\partial \Tilde{h}}{\partial t} (t) 
    = \mathcal{F}_{\chi_0, \frac{1}{2}}[\Tilde{h}(t)] + \mathcal{G}_{\chi_0}[v(t)].
\]
with 
\[
    v(t) = - \frac{1}{2}\int_0^t e^{-(t-s)} u(s) ds + u(t)
\]
\[
    \mathcal{F}_{\chi, \omega}[f](s) 
    = - \chi(x) \int_0^x \overline{\chi(\xi)} f(\xi) d \xi
    - \omega f
\]
\[
    \mathcal{G}_{\chi}[a](s) = a \chi(x) 
\]
\[
    \chi_0(x) = \sqrt{2x + 1}
\]
\end{screen}
この議論から導かれた作用素を\(\mathcal{F}_{\chi, \omega}, \mathcal{G}_{\chi}\)
を無限次元空間上のHiPPOに対応するものとして考えて、Infinite-dimensional HiPPO の定義とする。
\begin{screen}
\textcolor{blue}{\textbf{Definition:}}
\(\mathcal{F}_{\chi, \omega}, \mathcal{G}_{\chi}\)をInfinite-dimensional HiPPOという。
\end{screen}
このInfinite-dimensional HiPPOやそれを用いたLSSLについて、いくつかの表示が得られる。それを次に示す。

{
\small
(*3) 
\( 
    \mathcal{F}_{\chi, \omega} : L^2([0, N]) \to L^2([0, N])
\), 
\(
    \mathcal{G}_{\chi} : \mathbb{C} \to L^2([0, N])
\)に注意。
}


\end{block}

\end{column}

\begin{column}{0.48\linewidth}



\begin{block}{Infinite-dimensional HiPPOによるLSSLの解の積分表示}
Infinite-dimensional HiPPOに対して、$\exp(t\mathcal{F}_{\varphi, \omega})$は次のように積分を用いて表すことができる。
\begin{screen}
\textcolor{purple}{\textbf{Theorems:}}
    任意の$t \in \mathbb{R}_{>0}$に対し、
    \[
    \begin{split}
    & \exp\left({t\mathcal{F}_{\chi, \omega}}\right)[g](x) \\
    & = \exp(-c\omega) \left(g(x) + \chi(x) \int_0^x
    g(\xi) \overline{\chi(\xi)}
    \frac{\sqrt{t} J_1 
    \left({2\sqrt{t}\sqrt{\int_{\xi}^x \left|\chi(s)\right|^2 \mathrm{d} s }}\right)}
    {\sqrt{\int_{\xi}^x \left|\chi(s)\right|^2 \mathrm{d} s}}
    \mathrm{d} \xi\right)
    \end{split}
    \]
    特に$g = \chi$のとき 
    \begin{equation}        
    \begin{split}
        \exp\left({t\mathcal{F}_{\chi, \omega}}\right)[\chi](x) 
        = 
        \exp\left({-t\omega} \chi(x)\right)
        J_0
        \left({2\sqrt{t}\sqrt{\int_{a}^x \left|\chi(s)\right|^2 \mathrm{d} s}}\right) 
    \end{split}
    \end{equation}
    が成り立つ。
\end{screen}
従って、畳み込み核が
\[
    K(t) = e^{- \omega t} \int_0^N c(x) \chi(x) J_0\left(2\sqrt{t}\sqrt{\int_{0}^x \left|\chi(s)\right|^2 \mathrm{d} s}\right) \mathrm{d} x
\]
と明示的に書ける。
これを用いることで、Infinite-dimensional HiPPOによるLSSLの解が次のように明示的に求まる。
\begin{screen}
\textcolor{purple}{\textbf{Theorems:}}
    $v$はLipschitz連続であるとする。 
    
    このとき、Infinite-dimensional HiPPOを用いたLSSLの強解は次のように書ける。
    \[
        y(t) = \int_0^t \left(
        e^{- \omega (t - \tau)} \int_0^N c(x) \chi(x) J_0\left(2\sqrt{t - \tau}\sqrt{\int_{0}^x \left|\chi(s)\right|^2 \mathrm{d} s}\right) \mathrm{d} x
        \right) 
        v(\tau)  \mathrm{d} \tau 
    \]
\end{screen}

Remarks:

$c(x) = \Hat{c}\left({\sqrt{\int_{0}^x \left|\chi(s)\right|^2 \mathrm{d} s}}\right)\overline{\chi(x)}, 
\theta = \sqrt{\int_{0}^N \left|\chi(s)\right|^2 \mathrm{d} s}$
と置くと、
\[
    y(t) 
    = \int_0^t  
    2 \mathcal{H}_0\left[{
    \Hat{c} \cdot \mathbbm{1}_{[0, \theta]}
    }\right](2\sqrt{t - \tau}) 
    \cdot
    e^{- \omega (t - \tau)} \cdot v(\tau)  \mathrm{d} \tau.
\]
となる。ここで、$\mathcal{H}_0$は$0$次ハンケル変換である。この式では、$\chi$の自由度はパラメータ$\hat{c}$に吸収されていることがわかる。
\end{block}




\begin{block}{数値実験}

元のHiPPO行列を用いたS4model(ただし、HiPPO行列は固定)と、畳み込み核を上記の$K(t)$に置き換えたモデルを比較した結果を以下に示す。

\begin{figure}[h]
    \begin{minipage}{0.9\linewidth}
        \includegraphics[width=0.9\linewidth]
        {figure2.eps}
        \captionof{figure}{
            Sequential MNIST学習時の検証誤差の学習曲線
        }
    \end{minipage}
    
    \begin{minipage}{0.9\linewidth}
        \centering
        \input{tables}
    \end{minipage}
\end{figure}

この結果から、置き換えた後でも同等以上の性能が出ることがわかる。

\end{block}


\begin{block}{Conclusions}
\begin{itemize}
\item 有限次元のHIPPOを用いたLSSLは関数空間上での状態空間モデルの離散化であることを示した。
\item また、これを用いて無限次元HiPPOを提案し、これによるLSSLは明示的解を持つことを示した。
\item この明示的な形は、理論的な解析のしやすさや、高速に計算できるアルゴリズムの可能性を与えている。
\end{itemize}
\end{block}



\begin{block}{References}
\footnotesize
\begin{multicols}{2}  % 2カラムに設定
    \beamertemplatetextbibitems
    \bibliographystyle{plain}
    \bibliography{mybib}
\end{multicols}
\end{block}




\end{column}
\end{columns}





\end{frame}
\end{CJK}
\end{document}
